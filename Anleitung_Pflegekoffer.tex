%%%%%%%%%%%%%%%%%%%%%%%%%%%%%%%%%%%%%%%%%%%%%%%%
% COPYRIGHT: (C) 2018- FAU FabLab
%  CC-BY-SA 3.0
%%%%%%%%%%%%%%%%%%%%%%%%%%%%%%%%%%%%%%%%%%%%%%%%


\newcommand{\basedir}{fablab-document}
\documentclass{\basedir/fablab-document}
\linespread{1.2}

\author{zerspanung@fablab.fau.de}
\title{Anleitung KSS-Emulsions-Pflegekoffer}

\begin{document}

Der KSS-Emulsions-Pflegekoffer darf nur von Drehbank-Betreuern benutzt werden.

Lieferant: Hoffmann, Bestellnummer: 084310 \enquote{Emulsions-Pflegekoffer nach TRGS 611}

\section{KSS Werte messen}
die PH, Nitrat, Nitrid und Wasserhärte Werte mit den jeweiligen Messstreifen messen.
folgende Werte sollten eingehalten werden:
\begin{itemize}
\item PH-Wert: zwischen 8 und 10 (auf keinen Fall darf er sauer, d.h. kleiner, werden)
\item Nitrat Wert: <50 mg/l
\item Nitrid Wert: <20 mg/l
\item Wasserhärte: zwischen 10 und 15$^\circ$dH (niedriger ist nicht schlimm, kann dann aber schäumen)
\end{itemize}

\section{KSS Konzentration messen}
Den Refraktometer aus seiner Tasche nehmen, die Abdeckung hochklappen, mit der Pipette einen Tropfen KSS auf die blaue Scheibe tropfen und die Abdeckung wieder herunter klappen.
Mit einem Auge durch den Refraktometer gegen eine Lichtquelle schauen, es sollte jetzt eine klare Linie erkennbar sein, an dieser kann man die Konzentration des KSS Konzentrats ablesen. Die Konzentration sollte 5-10\% betragen.

Ist die Konzentration niedriger, so muss man etwas KSS Konzentrat hinzugeben (Konzentrat nicht direkt ins Becken, erst mit Wasser oder der bisherigen Emulsion mischen). Damit sich alles vollständig vermischt, die KSS-Pumpe länger laufen lassen und ggf. den KSS-Vorrat im Boden der Maschine umrühren. Anschließend erneut messen.

\section{Dokumentation}
Die Messwerte in den Wartungsplan der Drehbank eintragen. Sind die Werte außer der Toleranz, Drehbank außer Betrieb setzen (Schild anbringen \textbf{und} Email an zerspanung@fablab.fau.de) und Wechsel des KSS veranlassen.

\ccLicense{KSS-Emulsions-Pflegekoffer}{Anleitung KSS-Emulsions-Pflegekoffer}

\end{document}
